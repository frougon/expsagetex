\documentclass{article}
\usepackage{expsagetex}
\usepackage{amsmath}
\usepackage{parskip}
\usepackage{shortvrb}
\usepackage{csquotes}
\usepackage{expl3}
\usepackage[svgnames]{xcolor}
\usepackage{tikz}
\usepackage{xfp}
\usepackage{siunitx}
\sisetup{group-separator = \text{\,}}
\usepackage{hyperref}
\hypersetup{colorlinks=true, anchorcolor=Lime, linkcolor=RoyalBlue,
            urlcolor=Crimson}

\title{%
  \pkg{expsagetex} -- a fully expandable interface for SageTeX%
  \thanks{This file describes \pkg{expsagetex}~v0.1, last revised 2020-01-11.}%
}
\author{%
  Florent Rougon%
  \thanks{%
    E-mail: \href{mailto:f.rougon@free.fr}{mailto:f.rougon@free.fr}%
  }%
}

\newcommand*{\pkg}{\textsf}
\newcommand*{\meta}[1]{$\langle \textsl{#1} \rangle$}
\newcommand*{\tikzname}{Ti\emph{k}Z}

\setlength{\parindent}{0pt}     % will probably look better here

\begin{document}

\maketitle
\begin{abstract}
  This package is based on \texttt{sagetex.sty}, the style file for SageTeX.
  One deficiency of current \texttt{sagetex.sty} (version~3.3 from 2019/01/09)
  is that it doesn't provide any user-accessible way to retrieve a result
  computed by Sage in places where only expansion happens. This packages uses
  the low-level SageTeX machinery to implement a set of commands that can
  \enquote{record} results computed by Sage and a fully expandable command
  that can yield any such result wherever you need it, even in expansion-only
  contexts.
\end{abstract}

\MakeShortVerb{\|}

\tableofcontents
\clearpage

Besides the \nameref{sec:quick-start} section below, there is introductory
material to \pkg{expsagetex}
\href{https://tex.stackexchange.com/questions/521319/sagetex-1000sep-for-sage-calculated-number-siunitx/521389#521389}{on
  TeX.stackexchange.com}.

\section{Quick start}
\label{sec:quick-start}

Getting a result from Sage using \pkg{expsagetex} involves three steps in the
\LaTeX\ code:
\begin{enumerate}
\item Recording a Python expression to be written to the
  \texttt{.sagetex.sage} file. This step is done with one of the |\est*Record|
  functions.
\item Using the result where you need it. This is done with |\estGet| and
  works in many places, in particular in expansion-only contexts (inside
  |\edef|, |\message|, |\write|, etc.).
\item Declaring that you use the result, so that a warning can be displayed if
  the result isn't available yet (|\estGet| can't print the warning itself
  because it must work in expansion-only contexts). This is done with
  |\estRefUsed|.
\end{enumerate}

Step~3 can be done before or after step~2, but must come after step~1. When
you run |sage| on the \texttt{.sagetex.sage} file written by step~1, the
Python expression is evaluated, converted to a Python string and written to
the |.sagetex.sout| file. This file is in turn read by \texttt{sagetex.sty} at
startup, which makes the result available for |\estGet|.

Thus, a very simple example can be
\estRecordFormatted{\littleFermatExple}{(45**6) \percent 7}%
\estRefUsed{\littleFermatExple}%
$45^6 \equiv \estGet{\littleFermatExple} \pmod 7$. This was obtained with:
\begin{verbatim}
\estRecordFormatted{\littleFermatExple}{(45**6) \percent 7}%
\estRefUsed{\littleFermatExple}%
$45^6 \equiv \estGet{\littleFermatExple} \pmod 7$
\end{verbatim}

For compiling, use the same method as with \pkg{sagetex} (which
\pkg{expsagetex} relies on): you need one \LaTeX\ run, one |sage| run on the
\texttt{.sagetex.sage} file followed by at least one more \LaTeX\ run (if more
are needed, warnings are printed by \LaTeX\ as usual).

\section{Examples}

\subsection{Integers}

\subsubsection{Local assignments}

When using |\estRecordFormatted{|\meta{macro}|}{|\meta{expr}|}| with a Python
expression \meta{expr} that evaluates to an integer,
|\estGet[|\meta{fallback}|]{|\meta{macro}|}| should give (i.e., expand to) the
expected result. This is because calling |str()| on a Python integer doesn't
use any exponent notation, thus |latex()| should not add any formatting. Let's
execute the following Python statement using the |sageblock| environment:

\begin{sageblock}
x = 2**333
\end{sageblock}

Now, record the value of $x$ and associate it with macro |\mymacro|. This
makes |\mymacro| a kind of reference to the saved value, but to retrieve the
tokens associated to this reference, you have to use |\estGet| instead of
|\ref| (this can be done in an expansion-only context like inside |\edef|,
which is impossible with |\ref|). See section~\ref{sec:technical-details}
below for more details on this.
{%
  \estRecordFormatted{\mymacro}{x}%
  Recover the saved value and format it with the |\num| macro of
  \pkg{siunitx}:\\
  \num{\estGet[-1]{\mymacro}}.%
  \estRefUsed{\mymacro}\\
  This was achieved with the following calls:
\begin{verbatim}
\estRecordFormatted{\mymacro}{x}
\num{\estGet[-1]{\mymacro}}
\estRefUsed{\mymacro}
\end{verbatim}
}%
\ifx\mymacro\undefined\else
  \par Bug: |\mymacro| should be undefined here!
\fi

Another way to do the same without having Sage call |latex()| at all is to
use |\estRecordStr| to record a string representation of the integer we are
interested in:
\begin{verbatim}
\estRecordStr{\mymacro}{str(x)}
\num{\estGet[-1]{\mymacro}}
\estRefUsed{\mymacro}
\end{verbatim}

\pagebreak[3]

Notes:
\begin{itemize}
\item |\estGet| wraps its return value within |\unexpanded| (|\exp_not:n|),
  which implies that it won't expand further when used in |\edef| or an
  \pkg{expl3} |x|-type argument.
\item In order to use the result of evaluating a Python expression in the
  \LaTeX\ document, it is not necessary to store it in a Python variable.
  Example:%
  \estRecordFormatted{\twoToTheThenth}{2**10}%
  \estRefUsed{\twoToTheThenth}%
  \[ 2^{10} = \estGet{\twoToTheThenth} \]
\end{itemize}

Another example with variables:

\begin{sageblock}
x = 23 % 8         # That is, x = 7.
y = 2*x
\end{sageblock}

Save the value of $1000000\times (y-10)$ (as computed by Sage) and associate
it with macro |\numberBasedOnY|.
\estRecordFormatted{\numberBasedOnY}{1000000*(y-10)}%
% Use the value -1000000 when Sage hasn't written the result to the .sout file
% yet.
\[ 1000000\times (y - 10) = \num{\estGet[-1000000]{\numberBasedOnY}} \]
\estRefUsed{\numberBasedOnY}

\subsubsection{Global assignments}

While |\estRecordFormatted| assigns locally to the macro given by its
first argument, all |\est*Record| functions have a global variant that
performs a global assignment instead. The global variant of
|\estRecordFormatted| is |\estGRecordFormatted|, that of
|\estRecordStr| is |\estGRecordStr|, that of
|\estARecordFormatted| is |\estGARecordFormatted|, etc. Example:

\begin{sageblock}
x = 3+4
\end{sageblock}

Save the value of $x$ and globally assign macro |\Iamseven| to allow us
to retrieve the saved value.
{%
  \estGRecordFormatted{\Iamseven}{x}%
}%
Get it back from outside the group where this was done, and format it with
|\num|: the result is
\num{\estGet[-2]{\Iamseven}}.%
\estRefUsed{\Iamseven}

\subsection{Floating point numbers}

|\estRecordFormatted| works with floating point numbers, but it uses \LaTeX\
markup if the string representation of the result (in Python) uses exponent
notation. When this is not desirable---i.e., when you just want a numerical
result without any formatting---use |\estRecordFloat| or one of its variants.
Example:
\begin{sageblock}
x = 3.1415927
\end{sageblock}
\estRecordFloat{\PiRoundedByPython}{x}{.2f}%
\estRefUsed{\PiRoundedByPython}%
$x$ rounded to two decimal places is \estGet[0]{\PiRoundedByPython}. This was
rounded by Python.
\estRecordFloat{\PiRoundedBySiunitx}{x}{.10f}%
\estRefUsed{\PiRoundedBySiunitx}%
Thanks to the expandable nature of |\estGet|, one can also do this: $x$
rounded to four decimal places is
\num[round-mode=places, round-precision=4]{\estGet[0]{\PiRoundedBySiunitx}}.
This was rounded by \pkg{siunitx}.

Of course, one can also compute a float with Sage without storing it into a
Python variable. Here is an approximation of $\pi$ using an expression based
on Machin's formula and the power series expansion of $\arctan$:
\estRecordFloat{\PiViaMachinAndPowerSeries}
  {
    N(4*sum([ (-1)**n / (2*n+1) * (4*(1/5)**(2*n+1) - (1/239)**(2*n+1))
              for n in range(10) ]))
  }{.6f}%
\estRefUsed{\PiViaMachinAndPowerSeries}%
\begin{align*}
  \pi &= 4\sum_{n=0}^{+\infty} \frac{(-1)^n}{(2n+1)}
         \Biggl( 4 \, {\Bigl( \frac{1}{5} \Bigr)}^{2n+1} \! -
                 {\Bigl( \frac{1}{239} \Bigr)}^{2n+1} \Biggr)\\
      &\approx \estGet[0]{\PiViaMachinAndPowerSeries}
\end{align*}

\subsection{Strings}

Functions working on Python strings are the most general here. For instance,
|\estRecordFormatted| first evaluates |latex(...)| in Sage (Python), where
|...| is the Python expression you entered; this results in a string with
\LaTeX\ math-mode formatting commands (e.g., if evaluating |str(...)| in Python string yields something containing brackets or
exponents); then this string is written to the |.sout| file inside an
argument of |\newlabel|, which \pkg{sagetex} reads at startup. Similarly,
|\estRecordFloat| causes Python to convert a float to a string using the
user-specified format, then this string is written to the |.sout| file
inside an argument of |\newlabel|, just as with
|\estRecordFormatted|. This applies to global and array variants of the
\pkg{expsagetex} functions as well.

So, |\estRecordStr| and its variants can be very useful in case your
Python code returns a data type that isn't handled in a satisfactory way by
functions belonging to the families of |\estRecordFormatted| or
|\estRecordFloat|. All you have to do is to make your Python code format
your data as a string with a syntax that is convenient for the \LaTeX\
document; you record this string with |\estRecordStr| or one of its
variants, use |\estGet| and |\estRefUsed| as usual and voilà, your new data
type is nicely handled by \pkg{expsagetex}. We'll see an example of this
approach in
section~\ref{sec:plotting-with-tikzname-based-on-Sage-computations}, where
Python code generates a large amount of $(x_i, y_i)$ coordinates from two
lists of floats $[x_1, \dotsc, x_n]$ and $[y_1, \dotsc, y_n]$. The Python code
formats the list of $(x_i, y_i)$ coordinates in a form that is suitable for
the \tikzname\ \texttt{plot coordinates} operation, which allows one to plot
the data right away.

Here is a very simple example with |\estRecordStr| used to record a Sage
(Python) string:
\begin{sageblock}
 some_string = "abc  def"
\end{sageblock}
\estRecordStr{\somestring}{some_string}%
X\estGet{\somestring}Y%
\estRefUsed{\somestring}

The two spaces between |abc| and |def| are present in the
|.sout| file, but get coalesced into a single space token when
|\newlabel| tokenizes its second argument. If this is not desired, the
easiest way is probably to use another character instead of space (e.g.,
\verb|~|).

\subsection{Using array data}
\label{sec:using-array-data}

\pkg{expsagetex} offers variants of |\estRecordFormatted|,
|\estGRecordFormatted|, |\estRecordFloat|, |\estGRecordFloat|, |\estRecordStr|
and |\estGRecordStr| that allow one to easily use an index inside the macro
name given for the recording operation. These variants all have the letter
\texttt{A} (standing for ``array'') right before \texttt{Record}.

Where the non-array versions of the |\est*Record*| commands take one argument
specifying the destination macro name, the array versions accept two arguments
\meta{base} and \meta{index} and determine the destination macro name as the
result of expanding |\csname |\meta{base}@\meta{index}|\endcsname|. As a
consequence, the \meta{base} and \meta{index} arguments are recursively
expanded, which can be used to dynamically compute a numeric index using for
instance |\numexpr|, or whatever you want.

Here is an example that uses Python to evaluate the elements of a Vandermonde
matrix:
\ExplSyntaxOn
\int_new:N \l__my_row_int
\int_new:N \l__my_col_int
\int_new:N \g__my_row_counter_int
\clist_const:Nn \c__my_clist { 2, sqrt(2)/2, 2/7, 4/3, 1/5 }
\clist_const:Nn \c__my_formatted_clist
  { 2, \frac{\sqrt{2}}{2}, \frac{2}{7}, \frac{4}{3}, \frac{1}{5} }
\seq_new:N \l__my_matrix_rows_seq

\int_step_variable:nnNn { 1 } { 5 } \l__my_row_int
  {
    % One l3seq per matrix row
    \seq_clear_new:c { l__my_matrix_row \l__my_row_int _seq }
    % \use:c needs two expansion steps to do its work
    \exp_args:NNo \seq_put_right:No \l__my_matrix_rows_seq
      { \use:c { l__my_matrix_row \l__my_row_int _seq } }

    \int_step_variable:nnNn { 1 } { 5 } \l__my_col_int
      {
        \tl_set:Nx \l_tmpa_tl
          {
            ( \clist_item:Nn \c__my_clist { \l__my_row_int } ) **
                                                          ( \l__my_col_int - 1 )
          }
        \estARecordFormatted { Vandermonde } { \l__my_row_int - \l__my_col_int }
          { \tl_use:N \l_tmpa_tl }
        \seq_put_right:cx { l__my_matrix_row \l__my_row_int _seq }
          { \estAGet [0] { Vandermonde } { \l__my_row_int - \l__my_col_int } }
        \estARefUsed { Vandermonde } { \l__my_row_int - \l__my_col_int }
      }
  }

\[ \renewcommand{\arraystretch}{1.2}
  \mathcal{V} \Bigl( \clist_use:Nn \c__my_formatted_clist { , } \Bigr) =
   \begin{bmatrix}
     \int_gzero:N \g__my_row_counter_int
     \seq_map_inline:Nn \l__my_matrix_rows_seq
       {
         \int_gincr:N \g__my_row_counter_int
         \seq_use:Nn #1 { & }
         % Append \\ unless it is the last line
         \int_compare:nNnT
           { \g__my_row_counter_int } < { \seq_count:N \l__my_matrix_rows_seq }
           { \\ }
       }
   \end{bmatrix} \]
\ExplSyntaxOff

This technique can be used to plot curves with \tikzname\ based on coordinates
computed by Python, but this is rather wasteful. We'll see a better method
in section~\ref{sec:plotting-with-tikzname-based-on-Sage-computations}.

\subsection{Displaying lists}
\label{sec:lists}

|\estRecordFormatted| is fine for a list of integers:%
\estRecordFormatted{\listA}{list(range(2, 21, 3))}%
\begin{quote}
  \estRefUsed{\listA}%
  $\estGet[{[]}]{\listA}$
\end{quote}

It may do what you want for floats too:
\estRecordFormatted{\listB}{map(float, list(range(2, 21, 3)))}%
\begin{quote}
  \estRefUsed{\listB}%
  $\estGet[{[]}]{\listB}$
\end{quote}

But beware: if you use |\estRecordFormatted|, floats are formatted by Sage's
|latex()| function:
\estRecordFormatted{\listC}{[1e-10] + [5.0, 3e-14]}
\begin{quote}
  \small
  \estRefUsed{\listC}%
  $\estGet[{[]}]{\listC}$
\end{quote}

Here is a way to use a Python list of floats formatted with no exponent
and using a chosen number of decimal places:%
% We format each float individually as a string. We could as well omit the
% outer brackets: they are just string characters here (the final assembled
% expression is a Python string; it is *not* evaluated as a list by Python).
\estRecordStr{\listD}
  {
    '[' +
    ', '.join(map(lambda x: "\percent.14f" \percent (x,), [1e-10, 5.0, 3e-14])) +
    ']'
  }
\begin{quote}
  \estRefUsed{\listD}%
  $\estGet[{[]}]{\listD}$
\end{quote}

\subsection{Plotting with \tikzname\ based on Sage computations}
\label{sec:plotting-with-tikzname-based-on-Sage-computations}

Plotting functions based on coordinates computed by Sage can be done using
array-style functions of \pkg{expsagetex} (see
section~\ref{sec:using-array-data}), however such an approach is vastly
inefficient if the plot uses a significant number of points. A better approach
is to prepare one string on the Python side containing all needed coordinates
in a format that the \LaTeX\ plotting code can easily handle. This way, only
one reference---as in |\label| and |\ref|---is used to pass data for the whole
plot. As an example, here is a part of the Euler spiral where all coordinates
are computed by Sage and transferred in one go to the \LaTeX\ side (for the
full Euler Spiral, you need to make $t$ vary from $-\infty$ to $+\infty$).
%
\begin{sagesilent}
  x = var('x')
  start = -5*pi/2
  end = 5*pi/2
  nb_samples = 2000
  delta = float(end-start)/(nb_samples-1)
\end{sagesilent}
%
\estRecordStr{\computedCoordsRef}
  {
    " ".join([ "(\percent.6f, \percent.6f)" \percent
               (0.1*( numerical_integral(cos(x**2), 0, t)[0] ),
                0.1*( numerical_integral(sin(x**2), 0, t)[0] ))
               for t in ( start + _*delta for _ in range(nb_samples) ) ])
  }
\edef\computedCoords{\estGet[(0,0)]{\computedCoordsRef}}
\estRefUsed{\computedCoordsRef}
%
\begin{center}
\begin{tabular}{cc}
\raisebox{-0.5\height}{%
  \begin{tikzpicture}
    \draw[red!60!black, scale=20] plot coordinates { \computedCoords };
  \end{tikzpicture}%
}\hspace*{0.06\linewidth} &
\hspace*{0.06\linewidth}%
$ \newcommand{\diff}{\mathop{}\!\mathrm{d}}
  \left\{
    \begin{array}{>{\displaystyle}l}
      x(t) = \int_{0}^{t} \cos s^2 \diff s\\[10pt]
      y(t) = \int_{0}^{t} \sin s^2 \diff s
    \end{array}
  \right. ,\ t \in [-\frac{5\pi}{2}, \frac{5\pi}{2}]$
\end{tabular}
\end{center}

\section{Expandable commands work everywhere!}

Since |\estGet| is expandable and we have |\usepackage{xfp}| in the preamble,
we can do:
\begin{verbatim}
$\fpeval{3*\estGet[-1]{\Iamseven}}$
\end{verbatim}
This expands to $\fpeval{3*\estGet[-1]{\Iamseven}}$, since |\Iamseven|
contains the last value of $x$ computed by Sage, which is $7$ (if Sage hasn't
been run yet, it expands to $-3$ due to the fallback value \texttt{-1}
specified in the optional argument of |\estGet|).%
\estRefUsed{\Iamseven}%

{%
  Given that \TeX\ expands the right-hand side of integer, dimen and glue
  assignments until this yields the proper syntactic element, one can also use
  |\estGet| like this:
  \begin{verbatim}
  \count2=4
  \multiply \count2 by \estGet[-1]{\Iamseven}
  \the\count2
  \estRefUsed{\Iamseven}% just to be clean
  \end{verbatim}
  \count2=4
  \multiply \count2 by \estGet[-1]{\Iamseven}
  $\rightarrow$ \the\count2 % prints 28 (4*7)
  \estRefUsed{\Iamseven}
}

\bigskip
Same thing with a |\dimen| register:
\begin{verbatim}
\dimen0=1pt
\dimen0=\estGet[-1]{\Iamseven}\dimen0
\the\dimen0
\estRefUsed{\Iamseven}% just to be clean
\end{verbatim}
\dimen0=1pt
\dimen0=\estGet[-1]{\Iamseven}\dimen0
$\rightarrow$ \the\dimen0 % prints 7.0pt
\estRefUsed{\Iamseven}

\newcommand*{\grammarsymbol}[1]{\mbox{\meta{#1}}}%
And since \LaTeX\ lengths are |\skipdef| tokens and |\setlength| is
basically a glue assignment, one can have fun with
|\estGet[-1]{\Iamseven}| expanding to the \grammarsymbol{factor} of
a \grammarsymbol{normal dimen} (cf.~\TeX book p.~270):
\begin{verbatim}
\dimen0=2pt
\dimen2=3pt
\newlength{\mylength}
\setlength{\mylength}{%
       \estGet[-1]{\Iamseven}\dimen0
  plus \estGet[-1]{\Iamseven}\dimen2}%
\the\mylength
\estRefUsed{\Iamseven}% just to be clean
\end{verbatim}
\dimen0=2pt
\dimen2=3pt
\newlength{\mylength}
\setlength{\mylength}{%
       \estGet[-1]{\Iamseven}\dimen0
  plus \estGet[-1]{\Iamseven}\dimen2}%
$\rightarrow$ \the\mylength % prints 14.0pt plus 21.0pt
\estRefUsed{\Iamseven}

\section{Troubleshooting}

In case something goes wrong when using \pkg{sagetex} or \pkg{expsagetex}, you
may get stuck and unable to perform a full \LaTeX\ run. This can for instance
happen if one of your Python expressions is invalid. This kind of error is
similar to problems that may occur when something ``bad'' was written to the
|.aux| file. The cure is similar too: locate the error (inspecting the
\texttt{.sagetex.sage} and \texttt{.sagetex.sout} files can help; if the
problem is an invalid Python expression, the |sage| run will tell you),
regenerate the files containing invalid things after fixing their source, then
rerun the \LaTeX\ and |sage| commands as usual. Typically, you'll fix a Python
expression in your |.tex| file, remove the |.sagetex.sout| file and rerun
\LaTeX, |sage| and again \LaTeX. If you want to be \emph{really} sure to
restart from a clean state, remove the \texttt{.sagetex.sage},
\texttt{.sagetex.sout} and |.aux| files before redoing the \LaTeX\ \& |sage|
dance.

\section{Technical details}
\label{sec:technical-details}

Some comments about what a call such as |\estRecordStr{\mymacro}{|$s$|}| as
seen below really does ($s$ must evaluate in Python to a string). Informally,
one might be tempted to say that this saves the value of $s$ in the specified
macro, but that would be rather inaccurate. After the call to |\estRecordStr|,
the specified macro contains an integer which is used to form a reference
name---a label, if you wish---through which we can get the value of $s$ using
inner gears of the |\label| and |\ref| machinery. The \texttt{.sagetex.sout}
file written when running Sage on the \texttt{.sagetex.sage} file contains
|\newlabel| commands that define the associated text (tokens) for reference
names \texttt{@sageinline0}, \texttt{@sageinline1}, etc. The trailing number
is what is really stored by the above call to |\estRecordStr| as the
replacement text of |\mymacro|. \texttt{sagetex.sty} reads this
\texttt{.sagetex.sout} file at startup when it exists, which defines the
labels from the point of view of the \LaTeX\ kernel and allows us to retrieve
the Sage output for each recorded expression (after the |\newlabel| commands
have been executed, the Sage outputs are available, after some simple data
extraction, in macros |\r@@sageinline0|, |\r@@sageinline1|, etc.).

\end{document}
